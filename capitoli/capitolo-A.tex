% !TEX encoding = UTF-8
% !TEX TS-program = pdflatex
% !TEX root = ../tesi.tex
% !TEX spellcheck = it-IT

%**************************************************************
\chapter{Convenzioni tipografiche}
%**************************************************************

Riguardo la stesura del testo, relativamente al documento sono state adottate le seguenti convenzioni tipografiche:
\begin{itemize}
	\item Gli acronimi, le abbreviazioni e i termini ambigui o di uso non comune menzionati sono evidenziati dal colore blu e quindi definiti nel glossario, situato alla fine del presente documento;
	\item Tutti i riferimenti a capitoli, sezioni o figure presenti all'interno dell'elaborato sono anch'essi evidenziati in blu;
	\item I termini in lingua straniera o facenti parti del gergo tecnico sono marcati con il carattere corsivo;
	\item Le parole chiave di una determinata sezione sono scritte in grassetto;
	\item Negli elenchi puntati, per ogni punto, la prima parola presenta la prima lettera maiuscola;
    \item Negli elenchi puntati, in tutti i punti viene usato il punto e virgola per concludere, ad eccezione dell'ultimo che termina con un punto;
    \item Negli elenchi puntati quando si presenta un termine seguito dai due punti questo viene riportato in grassetto.
\end{itemize}

