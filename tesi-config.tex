%**************************************************************
% file contenente le impostazioni della tesi
%**************************************************************

%**************************************************************
% Frontespizio
%**************************************************************
\newcommand{\myName}{Marco Prelaz}                              % autore
\newcommand{\myTitle}{Supporto e consulenza tecnica dell'applicativo NPS}                    
\newcommand{\myDegree}{Tesi di laurea triennale}                % tipo di tesi
\newcommand{\myUni}{Università degli Studi di Padova}           % università
\newcommand{\myFaculty}{Corso di Laurea in Informatica}         % facoltà
\newcommand{\myDepartment}{Dipartimento di Matematica}          % dipartimento
\newcommand{\myProf}{Mauro Conti}                               % relatore
\newcommand{\myLocation}{Padova}                                % dove
\newcommand{\myAA}{2015-2016}                                   % anno accademico
\newcommand{\myTime}{Dicembre 2016}                             % quando


%**************************************************************
% Impostazioni di impaginazione
% see: http://wwwcdf.pd.infn.it/AppuntiLinux/a2547.htm
%**************************************************************

\setlength{\parindent}{0pt}   % larghezza rientro della prima riga (era 14pt)
\setlength{\parskip}{0pt}   % distanza tra i paragrafi


%**************************************************************
% Impostazioni di biblatex
%**************************************************************
\bibliography{bibliografia} % database di biblatex 

\defbibheading{bibliography}
{
    \cleardoublepage
    \phantomsection 
    \addcontentsline{toc}{chapter}{\bibname}
    \chapter*{\bibname\markboth{\bibname}{\bibname}}
}

\setlength\bibitemsep{1.5\itemsep} % spazio tra entry

%\DeclareBibliographyCategory{opere}
\DeclareBibliographyCategory{web}

%\addtocategory{opere}{womak:lean-thinking}
\addtocategory{web}{site:wikipedia}
\addtocategory{web}{site:mediana}
\addtocategory{web}{site:msdn}
\addtocategory{web}{site:closedxml}
\addtocategory{web}{site:nps}

%\defbibheading{opere}{\section*{Riferimenti bibliografici}}
\defbibheading{web}{\section*{Siti Web consultati}}


%**************************************************************
% Impostazioni di caption
%**************************************************************
\captionsetup{
    tableposition=top,
    figureposition=bottom,
    font=small,
    format=hang,
    labelfont=bf
}

%**************************************************************
% Impostazioni di glossaries
%**************************************************************

%**************************************************************
% Acronimi
%**************************************************************
\renewcommand{\acronymname}{Acronimi}

%\newacronym[description={\glslink{apig}{Application Program Interface}}]
%    {api}{API}{Application Program Interface}

%\newacronym[description={\glslink{umlg}{Unified Modeling Language}}]
%    {uml}{UML}{Unified Modeling Language}
    
\newacronym[description={\glslink{itg}{Information Technology}}]
    {it}{IT}{Information Technology}
    
\newacronym[description={\glslink{sfag}{Sales Force Automation}}]
    {sfa}{SFA}{Sales Force Automation}
    
\newacronym[description={\glslink{crg}{Change Request}}]
    {cr}{CR}{Change Request}
    
\newacronym[description={\glslink{ideg}{Integrated Development Environment}}]
    {ide}{IDE}{Integrated Development Environment}   
   
\newacronym[description={\glslink{aspg}{Active Server Pages}}]
    {asp}{ASP}{Active Server Pages} 
    
\newacronym[description={\glslink{clrg}{Common Language Runtime}}]
    {clr}{CLR}{Common Language Runtime}      
    
\newacronym[description={\glslink{domg}{Common Language Runtime}}]
    {dom}{DOM}{Document Object Model}         
    
\newacronym[description={\glslink{sdkg}{Software Development Kit}}]
    {sdk}{SDK}{Software Development Kit}        

%**************************************************************
% Glossario
%**************************************************************
\renewcommand{\glossaryname}{Glossario}

%\newglossaryentry{apig}
%{
%    name=\glslink{api}{API},
%    sort=api,
%    description={in informatica con il termine \emph{Application Programming Interface API} (ing. interfaccia di programmazione di un'applicazione) si indica ogni insieme di procedure disponibili al programmatore, di solito raggruppate a formare un set di strumenti specifici per l'espletamento di un determinato compito all'interno di un certo programma. La finalità è ottenere un'astrazione, di solito tra l'hardware e il programmatore o tra software a basso e quello ad alto livello semplificando così il lavoro di programmazione}
%}

%\newglossaryentry{umlg}
%{
%    name=\glslink{uml}{UML},
%    sort=uml,
%    description={in ingegneria del software \emph{UML, Unified Modeling Language} (ing. linguaggio di modellazione unificato) è un linguaggio di modellazione e specifica basato sul paradigma object-oriented. L'\emph{UML} svolge un'importantissima funzione di ``lingua franca'' nella comunità della progettazione e programmazione a oggetti. Gran parte della letteratura di settore usa tale linguaggio per descrivere soluzioni analitiche e progettuali in modo sintetico e comprensibile a un vasto pubblico}
%}

\newglossaryentry{sfag}
{
    name=\glslink{sfa}{Sales Force Automation},
    sort=Sales Force Automation,
    description={In italiano corrisponde all'Automazione della Forza di Vendita, ovvero tutti gli applicativi aziendali di supporto alle vendite. Questi programmano e controllano l'azione dei venditori, li assistono nella messa a punto di un piano di vendita o di promozione di un determinato prodotto e sussidiano la raccolta degli ordini dei clienti}
}

\newglossaryentry{itg}
{
    name=\glslink{it}{Information Technology},
    sort=Information Technology,
    description={In italiano Tecnologia dell'Informazione, indica l'utilizzo di elaboratori e attrezzature di telecomunicazione per memorizzare, recuperare, trasmettere
e manipolare dati, spesso nel contesto di un'attività commerciale o di un'altra
impresa}
}

\newglossaryentry{system integrator}
{
    name=\glslink{system integrator}{System Integrator},
    text=system integrator,
    sort=system integrator,
    description={In italiano Integratore di Sistemi, è una tipologia di carriera nell'ambito dell'\gls{it}. In particolare gli integratori di sistemi devono essere molto capaci di conformare i prodotti esistenti ai bisogni del cliente}
}

\newglossaryentry{customer experience}
{
    name=\glslink{customer experience}{Customer Experience},
    text=customer experience,
    sort=customer experience,
    description={Soddisfazione del cliente in base alla sua esperienza di fronte a qualsiasi contatto diretto o indiretto con un'azienda}
}

\newglossaryentry{crg}
{
    name=\glslink{cr}{Change Request},
    sort=change request,
    description={Modifica da apportare ad un applicativo rispetto ai requisiti espressi dal cliente in fase di analisi},
    plural=Change Requests
}

\newglossaryentry{wireframe}
{
    name=\glslink{wireframe}{Wireframe},
    sort=wireframe,
    description={Bozza strutturale di un sito, applicativo web o \textit{software} che permette di individuare subito le dinamiche del progetto in termini di usabilità e utilizzo pratico, i punti critici e quelli che richiedono uno sviluppo più accurato o solamente alcuni miglioramenti},
    plural=wireframes
}

\newglossaryentry{mockup}
{
    name=\glslink{mockup}{Mockup},
    text=mockup,
    sort=mockup,
    description={Modello statico di una pagina web o \textit{mobile} molto dettagliato che viene costruito mediante \textit{software} grafico e che dovrà quindi essere implementato in quello che sarà l'applicativo finale},
    plural=mockups
}

\newglossaryentry{deliverable}
{
    name=\glslink{deliverable}{Deliverable},
    text=deliverable,
    sort=deliverable,
    description={Indica un risultato verificabile, di significativa rilevanza, che deve essere conseguito durante un progetto e fornito al committente},
    plural=deliverables
}

\newglossaryentry{ios}
{
    name=\glslink{ios}{iOS},
    sort=ios,
    description={Sistema operativo sviluppato da \textit{Apple} per \textit{iPhone}, \textit{iPod touch} e \textit{iPad}}
}

\newglossaryentry{android}
{
    name=\glslink{android}{Android},
    sort=android,
    description={Sistema operativo per dispositivi \textit{mobili} sviluppato da \textit{Google Inc.} e basato sul \textit{kernel} \textit{Linux}}
}

\newglossaryentry{ideg}
{
    name=\glslink{ide}{Integrated Development Environment},
    sort=Integrated Development Environment,
    description={In italiano Ambiente di Sviluppo Integrato, è un \textit{software} che aiuta i programmatori nello sviluppo del codice sorgente di un applicativo, tramite una serie di strumenti e funzionalità dedicate}
}

\newglossaryentry{aspg}
{
    name=\glslink{asp}{Active Server Pages},
    sort=Active Server Pages,
    description={Pagine web contenenti, oltre al puro codice HTML, degli \textit{script} che verranno elaborati lato server per generare il codice HTML \textit{runtime} da inviare al \textit{browser} dell'utente. In questo modo è possibile mostrare contenuti dinamici (ad esempio estratti da \textit{database} che risiedono sul server web) e modificarne l'aspetto secondo le regole programmate negli \textit{script}, il tutto senza dover inviare il codice del programma all'utente finale (al quale va inviato solo il risultato), con notevole risparmio di tempi e banda}
}

\newglossaryentry{clrg}
{
    name=\glslink{clr}{Common Language Runtime},
    sort=Common Language Runtime,
    description={Ambiente di \textit{runtime} fornito dal \textit{framework} .NET, in cui viene eseguito il codice  e nel quale sono offerti servizi (come la gestione della memoria, dei \textit{thread} e dei servizi remoti) che facilitano il processo di sviluppo. La base è costituita da un compilatore JIT (Just In Time). Ciò significa che il codice intermedio prodotto, identico per tutti i linguaggi di alto livello impiegati (supportati dal \textit{framework} .NET), viene compilato in linguaggio macchina al momento della prima esecuzione}
}

\newglossaryentry{domg}
{
    name=\glslink{dom}{Document Object Model},
    sort=Document Object Model,
    description={In italiano Modello a Oggetti del Documento, è una forma di rappresentazione dei documenti strutturati come modello orientato agli oggetti}
}

\newglossaryentry{intranet}
{
    name=\glslink{intranet}{Intranet},
    text=intranet,
    sort=intranet,
    description={Rete aziendale privata accessibile solamente al personale di quell'organizzazione}
}

\newglossaryentry{data model}
{
    name=\glslink{data model}{Data Model},
    text=data model,
    sort=data model,
    description={Modello astratto che indica come si relazionano fra loro i dati all'interno di un sistema, per questo a volte viene anche definito come struttura dati}
}

\newglossaryentry{sistema di ticketing}
{
    name=\glslink{sistema di ticketing}{Sistema di ticketing},
    text=sistema di ticketing,
    sort=sistema di ticketing,
    description={Sistema informatico che gestisce e registra delle liste di richieste di assistenza o di problemi, organizzato secondo le necessità di chi offre il servizio}
}

\newglossaryentry{screenshot}
{
    name=\glslink{screenshot}{Screenshot},
    text=screenshot,
    sort=screenshot,
    description={Un'immagine che rappresenta lo stato dello schermo del computer in un determinato istante}
}

\newglossaryentry{sdkg}
{
    name=\glslink{sdk}{Software Development Kit},
    sort=software development kit,
    description={In italiano Pacchetto di Sviluppo per Applicazioni, indica genericamente un insieme di strumenti per lo sviluppo e la documentazione di \textit{software}}
}

\newglossaryentry{stored procedure}
{
    name=\glslink{stored procedure}{Stored procedure},
    text=stored procedure,
    sort=stored procedure,
    description={Gruppi di istruzioni SQL compattati in un modulo e memorizzati nel \textit{database} stesso per un successivo utilizzo}
}

\newglossaryentry{diagramma di Gantt}
{
    name=\glslink{diagramma di Gantt}{Diagramma di Gantt},
    text=diagramma di Gantt,
    sort=diagramma di Gantt,
    description={Strumento di supporto alla gestione dei progetti, che rappresenta graficamente un calendario delle attività dando una chiara illustrazione dello stato d'avanzamento del progetto raffigurato}
}

\newglossaryentry{deployment}
{
    name=\glslink{deployment}{Deployment},
    text=deployment,
    sort=deployment,
    description={Consegna o rilascio al cliente, con relativa installazione e messa
in funzione di una applicazione o di un sistema software tipicamente all'interno di un sistema informatico aziendale}
}
 % database di termini
\makeglossaries


%**************************************************************
% Impostazioni di graphicx
%**************************************************************
\graphicspath{{immagini/}} % cartella dove sono riposte le immagini


%**************************************************************
% Impostazioni di hyperref
%**************************************************************
\hypersetup{
    %hyperfootnotes=false,
    %pdfpagelabels,
    %draft, % = elimina tutti i link (utile per stampe in bianco e nero)
    colorlinks=true,
    linktocpage=true,
    pdfstartpage=1,
    pdfstartview=FitV,
    % decommenta la riga seguente per avere link in nero (per esempio per la stampa in bianco e nero)
    %colorlinks=false, linktocpage=false, pdfborder={0 0 0}, pdfstartpage=1, pdfstartview=FitV,
    breaklinks=true,
    pdfpagemode=UseNone,
    pageanchor=true,
    pdfpagemode=UseOutlines,
    plainpages=false,
    bookmarksnumbered,
    bookmarksopen=true,
    bookmarksopenlevel=1,
    hypertexnames=true,
    pdfhighlight=/O,
    %nesting=true,
    %frenchlinks,
    urlcolor=webbrown,
    linkcolor=RoyalBlue,
    citecolor=webgreen,
    %pagecolor=RoyalBlue,
    %urlcolor=Black, linkcolor=Black, citecolor=Black, %pagecolor=Black,
    pdftitle={\myTitle},
    pdfauthor={\textcopyright\ \myName, \myUni, \myFaculty},
    pdfsubject={},
    pdfkeywords={},
    pdfcreator={pdfLaTeX},
    pdfproducer={LaTeX}
}

%**************************************************************
% Impostazioni di itemize
%**************************************************************
%\renewcommand{\labelitemi}{$\ast$}

\renewcommand{\labelitemi}{$\bullet$}
%\renewcommand{\labelitemii}{$\cdot$}
%\renewcommand{\labelitemiii}{$\diamond$}
%\renewcommand{\labelitemiv}{$\ast$}


%**************************************************************
% Impostazioni di listings
%**************************************************************
\lstset{
    language=[LaTeX]Tex,%C++,
    keywordstyle=\color{RoyalBlue}, %\bfseries,
    basicstyle=\small\ttfamily,
    %identifierstyle=\color{NavyBlue},
    commentstyle=\color{Green}\ttfamily,
    stringstyle=\rmfamily,
    numbers=none, %left,%
    numberstyle=\scriptsize, %\tiny
    stepnumber=5,
    numbersep=8pt,
    showstringspaces=false,
    breaklines=true,
    frameround=ftff,
    frame=single
} 


%**************************************************************
% Impostazioni di xcolor
%**************************************************************
\definecolor{webgreen}{rgb}{0,.5,0}
\definecolor{webbrown}{rgb}{.6,0,0}


%**************************************************************
% Altro
%**************************************************************

\newcommand{\omissis}{[\dots\negthinspace]} % produce [...]

% eccezioni all'algoritmo di sillabazione
\hyphenation
{
    ma-cro-istru-zio-ne
    gi-ral-din
}

\newcommand{\sectionname}{sezione}
\addto\captionsitalian{\renewcommand{\figurename}{figura}
                       \renewcommand{\tablename}{tabella}}

\newcommand{\glsfirstoccur}{\ap{{[g]}}}

\newcommand{\intro}[1]{\emph{\textsf{#1}}}

%**************************************************************
% Environment per ``rischi''
%**************************************************************
\newcounter{riskcounter}                % define a counter
\setcounter{riskcounter}{0}             % set the counter to some initial value

%%%% Parameters
% #1: Title
\newenvironment{risk}[1]{
    \refstepcounter{riskcounter}        % increment counter
    \par \noindent                      % start new paragraph
    \textbf{\arabic{riskcounter}. #1}   % display the title before the 
                                        % content of the environment is displayed 
}{
    \par\medskip
}

\newcommand{\riskname}{Rischio}

\newcommand{\riskdescription}[1]{\textbf{\\Descrizione:} #1.}

\newcommand{\risksolution}[1]{\textbf{\\Soluzione:} #1.}

%**************************************************************
% Environment per ``use case''
%**************************************************************
\newcounter{usecasecounter}             % define a counter
\setcounter{usecasecounter}{0}          % set the counter to some initial value

%%%% Parameters
% #1: ID
% #2: Nome
\newenvironment{usecase}[2]{
    \renewcommand{\theusecasecounter}{\usecasename #1}  % this is where the display of 
                                                        % the counter is overwritten/modified
    \refstepcounter{usecasecounter}             % increment counter
    \vspace{10pt}
    \par \noindent                              % start new paragraph
    {\large \textbf{\usecasename #1: #2}}       % display the title before the 
                                                % content of the environment is displayed 
    \medskip
}{
    \medskip
}

\newcommand{\usecasename}{UC}

\newcommand{\usecaseactors}[1]{\textbf{\\Attori Principali:} #1. \vspace{4pt}}
\newcommand{\usecasepre}[1]{\textbf{\\Precondizioni:} #1. \vspace{4pt}}
\newcommand{\usecasedesc}[1]{\textbf{\\Descrizione:} #1. \vspace{4pt}}
\newcommand{\usecasepost}[1]{\textbf{\\Postcondizioni:} #1. \vspace{4pt}}
\newcommand{\usecasealt}[1]{\textbf{\\Scenario Alternativo:} #1. \vspace{4pt}}

%**************************************************************
% Environment per ``namespace description''
%**************************************************************

\newenvironment{namespacedesc}{
    \vspace{10pt}
    \par \noindent                              % start new paragraph
    \begin{description} 
}{
    \end{description}
    \medskip
}

\newcommand{\classdesc}[2]{\item[\textbf{#1:}] #2}