
%**************************************************************
% Acronimi
%**************************************************************
\renewcommand{\acronymname}{Acronimi}

%\newacronym[description={\glslink{apig}{Application Program Interface}}]
%    {api}{API}{Application Program Interface}

%\newacronym[description={\glslink{umlg}{Unified Modeling Language}}]
%    {uml}{UML}{Unified Modeling Language}
    
\newacronym[description={\glslink{itg}{Information Technology}}]
    {it}{IT}{Information Technology}
    
\newacronym[description={\glslink{sfag}{Sales Force Automation}}]
    {sfa}{SFA}{Sales Force Automation}
    
\newacronym[description={\glslink{crg}{Change Request}}]
    {cr}{CR}{Change Request}
    
\newacronym[description={\glslink{ideg}{Integrated Development Environment}}]
    {ide}{IDE}{Integrated Development Environment}   
   
\newacronym[description={\glslink{aspg}{Active Server Pages}}]
    {asp}{ASP}{Active Server Pages} 
    
\newacronym[description={\glslink{clrg}{Common Language Runtime}}]
    {clr}{CLR}{Common Language Runtime}      
    
\newacronym[description={\glslink{domg}{Common Language Runtime}}]
    {dom}{DOM}{Document Object Model}         
    
\newacronym[description={\glslink{sdkg}{Software Development Kit}}]
    {sdk}{SDK}{Software Development Kit}        

%**************************************************************
% Glossario
%**************************************************************
\renewcommand{\glossaryname}{Glossario}

%\newglossaryentry{apig}
%{
%    name=\glslink{api}{API},
%    sort=api,
%    description={in informatica con il termine \emph{Application Programming Interface API} (ing. interfaccia di programmazione di un'applicazione) si indica ogni insieme di procedure disponibili al programmatore, di solito raggruppate a formare un set di strumenti specifici per l'espletamento di un determinato compito all'interno di un certo programma. La finalità è ottenere un'astrazione, di solito tra l'hardware e il programmatore o tra software a basso e quello ad alto livello semplificando così il lavoro di programmazione}
%}

%\newglossaryentry{umlg}
%{
%    name=\glslink{uml}{UML},
%    sort=uml,
%    description={in ingegneria del software \emph{UML, Unified Modeling Language} (ing. linguaggio di modellazione unificato) è un linguaggio di modellazione e specifica basato sul paradigma object-oriented. L'\emph{UML} svolge un'importantissima funzione di ``lingua franca'' nella comunità della progettazione e programmazione a oggetti. Gran parte della letteratura di settore usa tale linguaggio per descrivere soluzioni analitiche e progettuali in modo sintetico e comprensibile a un vasto pubblico}
%}

\newglossaryentry{sfag}
{
    name=\glslink{sfa}{Sales Force Automation},
    sort=Sales Force Automation,
    description={In italiano corrisponde all'Automazione della Forza di Vendita, ovvero tutti gli applicativi aziendali di supporto alle vendite. Questi programmano e controllano l'azione dei venditori, li assistono nella messa a punto di un piano di vendita o di promozione di un determinato prodotto e sussidiano la raccolta degli ordini dei clienti}
}

\newglossaryentry{itg}
{
    name=\glslink{it}{Information Technology},
    sort=Information Technology,
    description={In italiano Tecnologia dell'Informazione, indica l'utilizzo di elaboratori e attrezzature di telecomunicazione per memorizzare, recuperare, trasmettere
e manipolare dati, spesso nel contesto di un'attività commerciale o di un'altra
impresa}
}

\newglossaryentry{system integrator}
{
    name=\glslink{system integrator}{System Integrator},
    text=system integrator,
    sort=system integrator,
    description={In italiano Integratore di Sistemi, è una tipologia di carriera nell'ambito dell'\gls{it}. In particolare gli integratori di sistemi devono essere molto capaci di conformare i prodotti esistenti ai bisogni del cliente}
}

\newglossaryentry{customer experience}
{
    name=\glslink{customer experience}{Customer Experience},
    text=customer experience,
    sort=customer experience,
    description={Soddisfazione del cliente in base alla sua esperienza di fronte a qualsiasi contatto diretto o indiretto con un'azienda}
}

\newglossaryentry{crg}
{
    name=\glslink{cr}{Change Request},
    sort=change request,
    description={Modifica da apportare ad un applicativo rispetto ai requisiti espressi dal cliente in fase di analisi},
    plural=Change Requests
}

\newglossaryentry{wireframe}
{
    name=\glslink{wireframe}{Wireframe},
    sort=wireframe,
    description={Bozza strutturale di un sito, applicativo web o \textit{software} che permette di individuare subito le dinamiche del progetto in termini di usabilità e utilizzo pratico, i punti critici e quelli che richiedono uno sviluppo più accurato o solamente alcuni miglioramenti},
    plural=wireframes
}

\newglossaryentry{mockup}
{
    name=\glslink{mockup}{Mockup},
    text=mockup,
    sort=mockup,
    description={Modello statico di una pagina web o \textit{mobile} molto dettagliato che viene costruito mediante \textit{software} grafico e che dovrà quindi essere implementato in quello che sarà l'applicativo finale},
    plural=mockups
}

\newglossaryentry{deliverable}
{
    name=\glslink{deliverable}{Deliverable},
    text=deliverable,
    sort=deliverable,
    description={Indica un risultato verificabile, di significativa rilevanza, che deve essere conseguito durante un progetto e fornito al committente},
    plural=deliverables
}

\newglossaryentry{ios}
{
    name=\glslink{ios}{iOS},
    sort=ios,
    description={Sistema operativo sviluppato da \textit{Apple} per \textit{iPhone}, \textit{iPod touch} e \textit{iPad}}
}

\newglossaryentry{android}
{
    name=\glslink{android}{Android},
    sort=android,
    description={Sistema operativo per dispositivi \textit{mobili} sviluppato da \textit{Google Inc.} e basato sul \textit{kernel} \textit{Linux}}
}

\newglossaryentry{ideg}
{
    name=\glslink{ide}{Integrated Development Environment},
    sort=Integrated Development Environment,
    description={In italiano Ambiente di Sviluppo Integrato, è un \textit{software} che aiuta i programmatori nello sviluppo del codice sorgente di un applicativo, tramite una serie di strumenti e funzionalità dedicate}
}

\newglossaryentry{aspg}
{
    name=\glslink{asp}{Active Server Pages},
    sort=Active Server Pages,
    description={Pagine web contenenti, oltre al puro codice HTML, degli \textit{script} che verranno elaborati lato server per generare il codice HTML \textit{runtime} da inviare al \textit{browser} dell'utente. In questo modo è possibile mostrare contenuti dinamici (ad esempio estratti da \textit{database} che risiedono sul server web) e modificarne l'aspetto secondo le regole programmate negli \textit{script}, il tutto senza dover inviare il codice del programma all'utente finale (al quale va inviato solo il risultato), con notevole risparmio di tempi e banda}
}

\newglossaryentry{clrg}
{
    name=\glslink{clr}{Common Language Runtime},
    sort=Common Language Runtime,
    description={Ambiente di \textit{runtime} fornito dal \textit{framework} .NET, in cui viene eseguito il codice  e nel quale sono offerti servizi (come la gestione della memoria, dei \textit{thread} e dei servizi remoti) che facilitano il processo di sviluppo. La base è costituita da un compilatore JIT (Just In Time). Ciò significa che il codice intermedio prodotto, identico per tutti i linguaggi di alto livello impiegati (supportati dal \textit{framework} .NET), viene compilato in linguaggio macchina al momento della prima esecuzione}
}

\newglossaryentry{domg}
{
    name=\glslink{dom}{Document Object Model},
    sort=Document Object Model,
    description={In italiano Modello a Oggetti del Documento, è una forma di rappresentazione dei documenti strutturati come modello orientato agli oggetti}
}

\newglossaryentry{intranet}
{
    name=\glslink{intranet}{Intranet},
    text=intranet,
    sort=intranet,
    description={Rete aziendale privata accessibile solamente al personale di quell'organizzazione}
}

\newglossaryentry{data model}
{
    name=\glslink{data model}{Data Model},
    text=data model,
    sort=data model,
    description={Modello astratto che indica come si relazionano fra loro i dati all'interno di un sistema, per questo a volte viene anche definito come struttura dati}
}

\newglossaryentry{sistema di ticketing}
{
    name=\glslink{sistema di ticketing}{Sistema di ticketing},
    text=sistema di ticketing,
    sort=sistema di ticketing,
    description={Sistema informatico che gestisce e registra delle liste di richieste di assistenza o di problemi, organizzato secondo le necessità di chi offre il servizio}
}

\newglossaryentry{screenshot}
{
    name=\glslink{screenshot}{Screenshot},
    text=screenshot,
    sort=screenshot,
    description={Un'immagine che rappresenta lo stato dello schermo del computer in un determinato istante}
}

\newglossaryentry{sdkg}
{
    name=\glslink{sdk}{Software Development Kit},
    sort=software development kit,
    description={In italiano Pacchetto di Sviluppo per Applicazioni, indica genericamente un insieme di strumenti per lo sviluppo e la documentazione di \textit{software}}
}

\newglossaryentry{stored procedure}
{
    name=\glslink{stored procedure}{Stored procedure},
    text=stored procedure,
    sort=stored procedure,
    description={Gruppi di istruzioni SQL compattati in un modulo e memorizzati nel \textit{database} stesso per un successivo utilizzo}
}

\newglossaryentry{diagramma di Gantt}
{
    name=\glslink{diagramma di Gantt}{Diagramma di Gantt},
    text=diagramma di Gantt,
    sort=diagramma di Gantt,
    description={Strumento di supporto alla gestione dei progetti, che rappresenta graficamente un calendario delle attività dando una chiara illustrazione dello stato d'avanzamento del progetto raffigurato}
}

\newglossaryentry{deployment}
{
    name=\glslink{deployment}{Deployment},
    text=deployment,
    sort=deployment,
    description={Consegna o rilascio al cliente, con relativa installazione e messa
in funzione di una applicazione o di un sistema software tipicamente all'interno di un sistema informatico aziendale}
}
